\documentclass[a4paper,12pt]{article}
\usepackage[T1]{fontenc}
\begin{document}
	\pagenumbering{arabic}%nums 1,2,3
	\pagestyle{empty}%non on front page
\title{Smart-Building Report}
\author{Simon Remington}
\date{\today}
\maketitle
\newpage
\tableofcontents
\newpage
\section{Introduction}
This is the report to for Smart Building Project. It covers the procedures I have taken so far.
\newline
This report is not a tutorial, nor does it cover all the rabbit holes and dead end to achieve my goal.
\section{Setting-up Raspberry Pi}
\subsection{Obtaining the Raspberry Pi Operating System}
The RPI image can be obtained from:
\newline
https://www.raspberrypi.org/downloads/raspbian/
\newline
The image is:
Raspbian Stretch Lite, a minimal image based on Debian Stretch.
The lite image is for a headless install. All communication to the RPI is made via ssh.
\newline
Version: November 2018
\newline
Release date: 2018-11-13
\newline
Kernel version: 4.14 
\newline
Download and unpack the image.
\newline
I have used both Windows and Linux Debian to create the bootable operating system so that I get a greater knowledge and understanding of completing the flash process.
%\newline
\subsection{Windows flashing}
Check the hash, SHA-256 of the image.
\newline
Use Windows built-in certUtil -hashfile Path/To/File/file.img SHA256 to compute the hash. Obtainable on both Windows 7 and 10 machines.
\newline
Insert the SD card into a Window's machine.
\newline
Use Etcher to flash image, obtained from:
\newline
$https://www.balena.io/etcher/$
\newline
Download Etcher and install.
\newline
Using Etcher, select source (Path to  .img), Destination (Tath to SD card)
Flash the card.
\newline
Dismount then remount SD card.
Donot repair the card when asked to by Windows.
\newline
CMD into the card then create an empty file named ssh with no extension, this can be accieved by entering the following command at the prompt:
\newline
C:\textbackslash \textgreater  type nul > ssh
\newline
% HOW TO BACKSLASH and other stuff
This will give secure shell access at boot time.
\newline
%\pagebreak
Info can be found at:
\newline
$https://www.raspberrypi.org/documentation/configuration/wireless/headless.md
$
\newline
\newline
Create a file named wpa\textunderscore,supplicant.conf
\newline
Info can be found at:
\newline
$https://manpages.debian.org/stretch/wpasupplicant/wpa_supplicant.conf.5.en.html/
$
\newline
\newline
$https://core-electronics.com.au/tutorials/raspberry-pi-zerow-headless-wifi-setup.html$
\newline
\newline
Add following to the file:
\newline
\newline
ctrl\textunderscore interface=DIR=/var/run/wpa\textunderscore supplicant GROUP=netdev
\newline
update\textunderscore
config=1
\newline
country=IE
\newline\newline
network=\{
\newline
\hspace*{10mm}ssid="Your-network"
\newline
\hspace*{10mm}key\textunderscore mgmt=NONE \# Use for no password
\newline\newline
network=\{
\newline
\hspace*{10mm}ssid="Your-network"
\newline
\hspace*{10mm}key\textunderscore psk="Your-password"

\}
\newline\newline
Now boot the Raspberry Pi, this SD card can be used in any machine that recognises the Debian Stretch OS.
\newline
This card if for the arm6 instruction set of the chip, it is capable of running in machines with the arm7 achitexture, but not the other way around. 
\subsection{Linux flashing}
Download Stretch.img and unpack.
\newline
\newline
Check hash at the prompt with:
\newline
user@host:\raisebox{-1ex}{\textasciitilde}\$ sha1sum /Path/To/file.img
\newline
\newline
Find the SD card.
\newline
user@host:\raisebox{-0.9ex}{\textasciitilde}\$ df -h
\newline
\newline
Dismount SD card.
\newline
user@host:\raisebox{-0.9ex}{\textasciitilde}\$ umount /dev/The SD card
\newline
\newline
Run the folliwing command:
\newline
user@host:\raisebox{-0.9ex}{\textasciitilde}\$ sudo dd bs=4M status=progress if=\raisebox{-1ex}{\textasciitilde}/Path/To/stretch-lite.img  of=/dev/The SD card
\newline
\newline
Info can be found at:
\newline
$https://elinux.org/RPi\textunderscore Easy\textunderscore SD\textunderscore Card\textunderscore Setup$
\newline
\newline
When SD card is written, mount the card and create files ssh, WPA
\newline
user@host:\raisebox{-0.9ex}{\textasciitilde}\$touch ssh
\newline
user@host:\raisebox{-0.9ex}{\textasciitilde}\$touch wpa\textunderscore supplicant.conf
\newline
\newline
Use an editor and create file contents as above for Windows.
\subsection{The Secure Shell}
Find ipaddress of Pi, look into router settings
\newline
Download and install Bitvise
Available at:
\newline
$https://www.bitvise.com/$
\newline
\newline
ssh to Raspberry Pi eg ssh pi@network address
\newline
change password from raspberry to your choice of passwrd
\newline
sudo raspi-config
\newline
expand file system
\newline
reboot
\newline\newline
apt update \&\& upgrade
\newline
\section{Docker}
Docker can be installed on the RPI with:
\newline\newline
\$curl -fsSL https://get.docker.com -o get-docker.sh
\newline
\$sudo sh get-docker.sh
\newline
Info at:
\newline
$https://docs.docker.com/install/linux/docker-ce/debian/\#install-using-the-convenience-script$
\newline\newline
Problem for Pi zero is that the latest docker is not friendly toward the arm6 with a segment error and did not work so I eventualy got it going by following this advice in the following forum.
\newline
\newline
$https://github.com/moby/moby/issues/38175$
\newline
\newline
at prompt:
\$apt-cache madison docker-ce
\newline
We can now see the packages
\newline
\$sudo apt-get purge docker-ce
\newline
\$sudo apt-get install docker-ce=<VERSION\textunderscore STRING>

\pagebreak
\section{this is section}
\subsection{this is sub}

\end{document}