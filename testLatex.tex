\documentclass[a4paper,12pt]{article}
\begin{document}
	\pagenumbering{arabic}%nums 1,2,3
	\pagestyle{empty}%non on front page
\title{Smart-Building Report}
\author{Simon Remington}
\date{\today}
\maketitle
\newpage

\tableofcontents
\newpage

\section{Introduction}
This is the report to my project. It covers the procedures I have taken so far
\section{Setting-up Raspberry Pi}
\subsection{Obtaining the Raspberry Pi Operating System}
The RPI image can be obtained from:
\newline
https://www.raspberrypi.org/downloads/raspbian/
\newline
The image is:
Raspbian Stretch Lite, a minimal image based on Debian Stretch.
The lite image is for a headless install. All communication to the RPI is made via ssh.
\newline
Version: November 2018
\newline
Release date: 2018-11-13
\newline
Kernel version: 4.14 
\newline
Download and unpack the image.
\newline
I have used both Windows and Linux Debian to create the bootable operating system so that I get a greater knowledge and understanding of completing the flash process.
%\newline
\subsection{Windows flashing}
Check the hash, SHA-256 of the image.
\newline
Use Windows built-in certUtil -hashfile Path/To/File/file.img SHA256 to compute hash
Insert SD card into Windows machine.
Use Etcher to flash image obtained from: https://www.balena.io/etcher/
Download Etcher and install.
\newline
from Etcher, select source (Stretch-lite img), Destination (SD card)
Flash the card
\subsection{Linux flashing}
now go to debian machine
\newpage
\section{New Paragraph}
well hello there
\pagebreak
\section{this is section}
\subsection{this is sub}

\end{document}